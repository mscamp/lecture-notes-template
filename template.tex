% Document Class
\documentclass[11pt]{report}

% Packages
\usepackage[italian]{babel}
\usepackage[utf8]{inputenc}
\usepackage{amsmath,amsthm,amsfonts,amssymb,amscd}
\usepackage{multirow,booktabs}
\usepackage[table]{xcolor}
\usepackage{fullpage}
\usepackage{lastpage}
\usepackage{enumitem}
\usepackage{fancyhdr}
\usepackage{mathrsfs}
\usepackage{wrapfig}
\usepackage{setspace}
\usepackage{calc}
\usepackage{multicol}
\usepackage{cancel}
\usepackage[margin=3cm]{geometry}
\usepackage{amsmath}
\newlength{\tabcont}
\setlength{\parindent}{0.0in}
\setlength{\parskip}{0.05in}
\usepackage{empheq}
\usepackage{framed}
\usepackage[most]{tcolorbox}
\usepackage{xcolor}
\usepackage{graphicx}
\graphicspath{{pictures/}}

% For braket notation
\usepackage{mathtools}
\DeclarePairedDelimiter\bra{\langle}{\rvert}
\DeclarePairedDelimiter\ket{\lvert}{\rangle}
\DeclarePairedDelimiterX\braket[2]{\langle}{\rangle}{#1\,\delimsize\vert\,\mathopen{}#2} 

% Styling
\colorlet{shadecolor}{blue!15}
\parindent 0in
\parskip 12pt
\geometry{margin=1in, headsep=0.25in}
\theoremstyle{definition}
\newtheorem{defn}{Definition}
\newtheorem{reg}{Rule}
\newtheorem{exer}{Exercise}
\newtheorem{note}{Note}
\setcounter{tocdepth}{2}
\renewcommand{\vec}[1]{\mathbf{#1}}

% Beginning of document
\begin{document}

% Header
 \begin{titlepage}
		\noindent
		\begin{minipage}[t]{0.19\textwidth}
			\vspace{-4mm}{\includegraphics[scale=1.15]{logo_unimib.pdf}}
		\end{minipage}
		\begin{minipage}[t]{0.81\textwidth}
			{
				\setstretch{1.42}
				{\textsc{Università degli Studi di Milano - Bicocca}} \\
				\textbf{Scuola di Scienze} \\
				\textbf{Dipartimento di Fisica} \\
				\textbf{Corso di Laurea Triennale in Fisica} \\
				\par
			}
		\end{minipage}
	
		\vspace{60mm}
		
		\begin{center}
			{\LARGE{
					\setstretch{1.2}
					\textbf{Corso \\ Nome Cognome}
					\par
			}}
		\end{center}
		
	    \vspace{50mm}
\end{titlepage}

% Table of contents
\tableofcontents

% Content
% Chapter title
\chapter{Titolo Lezione - Data}

% Content
\section{Section}
\subsection{Subsection}
Consider an inertial reference frame (i.e not accelerating) which will be denoted S$_0$, and a accelerating reference frame, \textit{S} that has an acceleration of \textit{A}. 

\begin{note}
\textbf{Capital Letters refer to the accelerating reference frame \textit{S} while lowercase letters refer to the inertial reference frame S$_0$}
\end{note}

\begin{shaded}
\textbf{Schrodinger's Cat} \\
\begin{equation}
\hat{H} \ket{\Psi} = E \ket{\Psi}
\end{equation}
\end{shaded}

\newpage


\end{document}
% End of document
